% HW3 high dimensional data

\documentclass[12pt, leqno]{article}
\usepackage{amsfonts, amsmath, amssymb}
\usepackage{amsthm}
\usepackage{mathtools}
\usepackage{fancyhdr}
\usepackage{hyperref}
\usepackage{graphicx}
\usepackage{caption}
\usepackage{subcaption}
\usepackage{float}
\usepackage{mathrsfs}
\usepackage{array} 
\usepackage{rotating}
\usepackage{rotating}
\usepackage{booktabs}
\usepackage{bbm}
%\usepackage{babel}
\providecommand{\abs}[1]{\lvert#1\rvert}
\providecommand{\norm}[1]{\lVert#1\rVert}
\newcommand{\macheps}{\epsilon_{\mbox{\scriptsize mach}}}
\let\oldhat\hat
\renewcommand{\vec}[1]{\mathbf{#1}}
\renewcommand{\hat}[1]{\oldhat{{#1}}}
\def\rp{\ensuremath \mathbb{R}^p}
\def\rpp{\ensuremath \mathbb{R}^{p \times p}}
\def\s{\ensuremath\Sigma}
\def\om{\ensuremath\Omega}
\def\pd{\ensuremath\mathbb{P}^+}
\def\pg{\ensuremath\mathbb{P}_{{G}}}
\def\E{\ensuremath\mathbb{E}}
\def\normdist[#1]#2{\ensuremath \sim \mathcal{N} (#1,#2) }
\def\ndist1{\ensuremath \sim \mathcal{N}  (\mu, \sigma)}
\def\ndistvec{\ensuremath \sim \mathcal{N}_p ( {\mu},  {\Sigma})}
\def\lra{\ensuremath\Leftrightarrow}
\def\stackrel#1#2{\mathrel{\mathop{#2}\limits^{#1}}}
\newcommand\ind{\protect\mathpalette{\protect\independenT}{\perp}}
\def\independenT#1#2{\mathrel{\rlap{$#1#2$}\mkern2mu{#1#2}}}
\makeatletter
\newtheorem{thm}{Theorem}[]
\newtheorem{lemma}{Lemma}[]
\newtheorem{defn}[thm]{Definition}
\newcommand{\sign}{\mathrm{sign}}
\newcommand{\distas}[1]{\mathbin{\overset{#1}{\kern\z@\sim}}}%
\newsavebox{\mybox}\newsavebox{\mysim}
\newcommand{\dist}[1]{%
  \savebox{\mybox}{\hbox{\kern3pt$\scriptstyle#1$\kern3pt}}%
  \savebox{\mysim}{\hbox{$\sim$}}%
  \mathbin{\overset{#1}{\kern\z@\resizebox{\wd\mybox}{\ht\mysim}{$\sim$}}}%
}
\makeatother

\begin{document}
\pagestyle{fancy}
\lhead{Syed Rahman}
\rhead{STA6707}

\begin{center}
{\large {\bf Homework 2 Partial Solutions}}
\end{center}

\paragraph{Problem 1:} For this problem, I was basically looking to
see whether everyone just did a proper cross-validation to find the
error rates, and that people weren't mixed up between apparent and
true error rates. Comparions for 5 classifers are shown below. For
$k-nn, k = 1$ gave the best results. 

 \begin{table}[ht]
\centering
\begin{tabular}{rr}
  \hline
 & Accuracy \\ 
  \hline
RANDOM FORESTS & 0.99 \\ 
  LDA & 0.99 \\ 
  QDA & 0.99 \\ 
  KNN & 0.77 \\ 
  DECISION TREES & 0.88 \\ 
   \hline
\end{tabular}
\end{table}

\newpage

\paragraph{Problem 2:} For this problem, we follow a similar procedure
using 10-fold cross validation (although the sample size is large
enough to avoid such a method). Comparions for 5 classifers are shown below. For
$k-nn, k = 1$ gave the best results. 

Random Forests give the best results for Region. Qda seems to be the best for Area. At this point we could simplt use random forests for predicting
regions and Qda for predicting Area, but we adopt a different
approach. For our predictions, we use the most common one amongst all
classifiers. There is one inconsistency between area/region, which is
observation six. 

\begin{table}[ht]
\centering
\begin{tabular}{rr}
  \hline
 & Accuracy \\ 
  \hline
RANDOM FORESTS & 1.00 \\ 
  LDA & 0.99 \\ 
  QDA & 1.00 \\ 
  KNN & 0.97 \\ 
  DECISION TREES & 0.99 \\ 
   \hline
\end{tabular}
\caption{Accuracy for Region}
\end{table}

% latex table generated in R 3.1.2 by xtable 1.7-4 package
% Thu Mar 17 16:35:14 2016
\begin{table}[ht]
\centering
\begin{tabular}{rr}
  \hline
 & Accuracy \\ 
  \hline
RANDOM FORESTS & 0.95 \\ 
  LDA & 0.94 \\ 
  QDA & 0.97 \\ 
  KNN & 0.92 \\ 
  DECISION TREES & 0.90 \\ 
   \hline
\end{tabular}
\caption{Accuracty for Area}
\end{table}

\begin{table}[ht]
\centering
\resizebox{\textwidth}{!}{
\begin{tabular}{rrrrrrrrrrrrrrrrrrrrr}
  \hline
 & 1 & 2 & 3 & 4 & 5 & 6 & 7 & 8 & 9 & 10 & 11 & 12 & 13 & 14 & 15 & 16 & 17 & 18 & 19 & 20 \\ 
  \hline
RANDOM FORESTS & 3.00 & 1.00 & 2.00 & 1.00 & 3.00 & 1.00 & 1.00 & 1.00 & 3.00 & 1.00 & 2.00 & 3.00 & 3.00 & 3.00 & 1.00 & 1.00 & 1.00 & 3.00 & 1.00 & 3.00 \\ 
  LDA & 3.00 & 1.00 & 2.00 & 1.00 & 3.00 & 3.00 & 1.00 & 3.00 & 3.00 & 1.00 & 2.00 & 3.00 & 2.00 & 3.00 & 1.00 & 1.00 & 3.00 & 3.00 & 1.00 & 3.00 \\ 
  QDA & 1.00 & 1.00 & 2.00 & 1.00 & 3.00 & 1.00 & 1.00 & 1.00 & 1.00 & 1.00 & 2.00 & 2.00 & 3.00 & 2.00 & 1.00 & 1.00 & 1.00 & 3.00 & 1.00 & 3.00 \\ 
  KNN & 3.00 & 1.00 & 2.00 & 1.00 & 3.00 & 3.00 & 1.00 & 1.00 & 3.00 & 1.00 & 2.00 & 3.00 & 3.00 & 3.00 & 1.00 & 1.00 & 3.00 & 3.00 & 1.00 & 3.00 \\ 
  DECISION TREES & 3.00 & 1.00 & 3.00 & 1.00 & 3.00 & 1.00 & 1.00 & 1.00 & 3.00 & 1.00 & 3.00 & 3.00 & 3.00 & 3.00 & 1.00 & 1.00 & 1.00 & 3.00 & 1.00 & 3.00 \\ 
  MODE & 3.00 & 1.00 & 2.00 & 1.00 & 3.00 & 1.00 & 1.00 & 1.00 & 3.00 & 1.00 & 2.00 & 3.00 & 3.00 & 3.00 & 1.00 & 1.00 & 1.00 & 3.00 & 1.00 & 3.00 \\ 
   \hline
\end{tabular}
}
\caption{Predictions for region}
\end{table}

% latex table generated in R 3.1.2 by xtable 1.7-4 package
% Thu Mar 17 16:53:41 2016
\begin{table}[ht]
\centering
\resizebox{\textwidth}{!}{
\begin{tabular}{rrrrrrrrrrrrrrrrrrrrr}
  \hline
 & 1 & 2 & 3 & 4 & 5 & 6 & 7 & 8 & 9 & 10 & 11 & 12 & 13 & 14 & 15 & 16 & 17 & 18 & 19 & 20 \\ 
  \hline
RANDOM FORESTS & 8.00 & 3.00 & 5.00 & 4.00 & 7.00 & 9.00 & 4.00 & 1.00 & 9.00 & 3.00 & 6.00 & 9.00 & 8.00 & 9.00 & 3.00 & 3.00 & 3.00 & 8.00 & 3.00 & 7.00 \\ 
  LDA & 8.00 & 2.00 & 5.00 & 3.00 & 7.00 & 9.00 & 2.00 & 9.00 & 7.00 & 3.00 & 9.00 & 9.00 & 8.00 & 7.00 & 2.00 & 2.00 & 8.00 & 8.00 & 3.00 & 7.00 \\ 
  QDA & 8.00 & 4.00 & 5.00 & 3.00 & 7.00 & 2.00 & 1.00 & 3.00 & 8.00 & 3.00 & 5.00 & 8.00 & 8.00 & 8.00 & 3.00 & 4.00 & 5.00 & 8.00 & 3.00 & 7.00 \\ 
  KNN & 8.00 & 3.00 & 5.00 & 4.00 & 7.00 & 9.00 & 3.00 & 1.00 & 9.00 & 3.00 & 5.00 & 9.00 & 8.00 & 9.00 & 3.00 & 3.00 & 8.00 & 8.00 & 3.00 & 7.00 \\ 
  DECISION TREES & 7.00 & 3.00 & 7.00 & 2.00 & 7.00 & 1.00 & 3.00 & 1.00 & 9.00 & 5.00 & 7.00 & 9.00 & 8.00 & 9.00 & 3.00 & 3.00 & 3.00 & 8.00 & 3.00 & 7.00 \\ 
  MODE & 8.00 & 3.00 & 5.00 & 4.00 & 7.00 & 9.00 & 3.00 & 1.00 & 9.00 & 3.00 & 5.00 & 9.00 & 8.00 & 9.00 & 3.00 & 3.00 & 3.00 & 8.00 & 3.00 & 7.00 \\ 
   \hline
\end{tabular}
}
\caption{Predictions for area}
\end{table}



 



\end{document}